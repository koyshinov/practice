В процессе прохождения производственной практики были получены навыки программирования с системой компьютерного зрения, навыки программирования микроконтроллеров arduino и передача информации по com-порту.

В ходе работы над проектом <<робота-повара>> были поняты некоторые недостатки и преимущества использованных библиотек и технологий.

Над данным проектом работало нескольно человек, поэтому были повышены навыки программирования в команде, чтение чужого программного кода. Также понята необходимость написание программ так, чтобы его могли прочитать другие сотрудники.