Машинное зрение — это применение компьютерного зрения для промышленности и производства. В то время как компьютерное зрение — это общий набор методов, позволяющих компьютерам видеть. Областью интереса машинного зрения, как инженерного направления, являются цифровые устройства ввода-вывода и компьютерные сети, предназначенные для контроля производственного оборудования, таких как роботы-манипуляторы или аппараты для извлечения бракованной продукции. Машинное зрение является подразделом инженерии, связанное с вычислительной техникой, оптикой, машиностроением и промышленной автоматизацией.[2]

Компьютеры не могут «видеть» таким же образом, как это делает человек. Фотокамеры не эквивалентны системе зрения человека, и в то время как люди могут опираться на догадки и предположения, системы машинного зрения должны «видеть» путём изучения отдельных пикселей изображения, обрабатывая их и пытаясь сделать выводы с помощью базы знаний и набора функций таких, как устройство распознавания образов. Хотя некоторые алгоритмы машинного зрения были разработаны, чтобы имитировать зрительное восприятие человека, большое количество уникальных методов были разработаны для обработки изображений и определения соответствующих свойств изображения.

Изображение с камеры попадает в захватчик кадров или в память компьютера в системах, где захватчик кадров не используется. 

Захватчик кадров — это устройство оцифровки (как часть умной камеры или в виде отдельной платы в компьютере), которое преобразует выходные данные с камеры в цифровой формат (как правило, это двумерный массив чисел, соответствующих уровню интенсивности света определенной точки в области зрения, называемых пикселями) и размещает изображения в памяти компьютера, так чтобы оно могло быть обработано с помощью программного обеспечения для машинного зрения.

Программное обеспечение, как правило, совершает несколько шагов для обработки изображений. Часто изображение для начала обрабатывается с целью уменьшения шума или конвертации множества оттенков серого в простое сочетание черного и белого (бинаризации). После первоначальной обработки программа будет считать, производить измерения и/или определять объекты, размеры, дефекты и другие характеристики изображения.

OpenCV (англ. Open Source Computer Vision Library, библиотека компьютерного зрения с открытым исходным кодом) — библиотека алгоритмов компьютерного зрения, обработки изображений и численных алгоритмов общего назначения с открытым кодом. Реализована на C/C++, также разрабатывается для Python, Java, Ruby, Matlab, Lua и других языков. Может свободно использоваться в академических и коммерческих целях — распространяется в условиях лицензии BSD.[3]

Модули библиотеки:
\begin{itemize}
  \item opencv\_core — основная функциональность. Включает в себя базовые структуры, вычисления (математические функции, генераторы случайных чисел) и линейную алгебру, ввод/вывод для XML и YAML и т. д;
  \item opencv\_imgproc — обработка изображений (фильтрация, геометрические преобразования, преобразование цветовых пространств и т. д.);
  \item opencv\_highgui — простой UI, ввод/вывод изображений и видео;
  \item opencv\_ml — модели машинного обучения (SVM, деревья решений, обучение со стимулированием и т. д.);
  \item opencv\_features2d — распознавание и описание плоских примитивов.
  \item opencv\_video — анализ движения и отслеживание объектов (оптический поток, шаблоны движения, устранение фона).
  \item opencv\_objdetect — обнаружение объектов на изображении (нахождение лиц с помощью алгоритма Виолы-Джонса (англ.), распознавание людей HOG и т. д.);
  \item opencv\_calib3d — калибровка камеры, поиск стерео-соответствия и элементы обработки трёхмерных данных;
  \item opencv\_flann — библиотека быстрого поиска ближайших соседей (FLANN 1.5) и обертки OpenCV;
  \item opencv\_gpu — ускорение некоторых функций OpenCV за счет CUDA, создан при поддержке NVidia.
\end{itemize}
