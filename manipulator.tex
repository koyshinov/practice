Манипулятор – совокупность пространственного рычажного механизма и системы приводов, осуществляющая под управлением программируемого автоматического устройства или человека-оператора действия (манипуляции), аналогичные действиям руки человека.

Промышленные роботы предназначены для замены человека при выполнении основных и вспомогательных технологических операций в процессе промышленного производства. При этом решается важная социальная задача -- освобождения человека от работ, связанных с опасностями для здоровья или с тяжелым физическим трудом, а также от простых монотонных операций, не требующих высокой квалификации. Гибкие автоматизированные производства, создаваемые на базе промышленных роботов, позволяют решать задачи автоматизации на предприятиях с широкой номенклатурой продукции при мелкосерийном и штучном производстве. Промышленные роботы являются важными составными частями современного промышленного производства.[1]

Манипулятор по принципу действия напоминает человеческую руку. В нём присутствуют поворотные соединения, которые обеспечивают наклон в плечевом соединении и сгибание в локте, механический захват, который позволит роботу хватать и перемещать предметы в разных направлениях.

Разрабатываемый в данной работе манипулятор будет иметь следующие параметры классификации:

\begin{itemize}
\item степень универсальности – специальный, предназначен для переноски небольших лёгких предметов;
\item тип приводов – электрический;
\item грузоподъемность – сверхлёгкий (до 1 кг);
\item подвижность робота – стационарный;
\item способ размещения – напольный;
\item способ управления – программный и ручной.
\end{itemize}