\documentclass[14pt]{eskdtext}
%\usepackage[nooneline]{caption} \captionsetup[table]{justification=raggedright} \captionsetup[figure]{justification=centering,labelsep=endash}
\usepackage[numbertop, numbercenter]{eskdplain}
\usepackage[utf8x]{inputenc}
\usepackage{listings}
% - Подключаем шрифты из пакета scalable-cyrfonts-tex
\usepackage{cyrtimes}

% - Отступ красной строки
\setlength{\parindent}{1.25cm}

% - Убирает точку в списке литературы
\makeatletter
\def\@biblabel#1{#1 }

% - Точки для всех пунктов в оглавлении
\renewcommand*{\l@section}{\@dottedtocline{1}{1.5em}{1.2em}}
\renewcommand*{\l@subsection}{\@dottedtocline{1}{2.3em}{1.9em}}
\renewcommand*{\l@subsubsection}{\@dottedtocline{1}{3.1em}{1.8em}}

% - Для переопределения списков
\renewcommand{\theenumi}{\arabic{enumi}}
\renewcommand{\labelenumi}{\theenumi)}
\makeatother

\usepackage{enumitem}
\setlist{nolistsep, itemsep=0.3cm,parsep=0pt}

% - ГОСТ списка литературы
\bibliographystyle{utf8gost705u}

% - Верикальные отступы заголовков 
\ESKDsectSkip{section}{1em}{1em}
\ESKDsectSkip{subsection}{1em}{1em}
\ESKDsectSkip{subsubsection}{1em}{1em}

% - Изменение заголовков
\usepackage{titlesec}
\titleformat{\section}{\centering\normalfont\normalsize}{\thesection}{1.0em}{}
\titleformat{\subsection}{\centering\normalfont\normalsize}{\thesubsection}{1.0em}{}
\titleformat{\subsubsection}{\centering\normalfont\normalsize}{\thesubsubsection}{1.0em}{}
\titleformat{\paragraph}{\centering\normalsize}{\theparagraph}{1.0em}{}

% - Оставим место под ТЗ 
%\setcounter{page}{4}

% - Для больших таблиц
\usepackage{longtable}
\usepackage{tabularx}
\renewcommand{\thetable}{\thesection.\arabic{table}}

% - Используем графику в документе
\usepackage{graphicx}
\graphicspath{{images/}}
\renewcommand{\thefigure}{\thesection.\arabic{figure}}

% - Счётчики
\usepackage{eskdtotal}

% - Выравнивание по ширине
\sloppy

% - запрет переноса слов
\hyphenpenalty=10000 

\RequirePackage{enumitem}
\renewcommand{\alph}[1]{\asbuk{#1}}
\renewcommand{\labelenumi}{\arabic{enumi}}% Меняем везде перечисления на цифра.цифра
\renewcommand{\labelenumii}{\arabic{enumi}.\arabic{enumii}}% Меняем везде перечисления на цифра.цифра
\setlist{nolistsep}
\setitemize[1]{label=--, fullwidth, itemindent=\parindent,
  listparindent=\parindent}% для дефисного списка
\setenumerate[1]{fullwidth, itemindent=\parindent, 
  listparindent=\parindent}% для нумерованного списка
\setenumerate[2]{fullwidth,  
  listparindent=\parindent, leftmargin=\parindent}% для списка 2-ой ступени

% - Оформляем листинг кода (не использовать комментарии на русском!)
\usepackage{listings}  
\lstset{basicstyle=\ttfamily\small}
\lstset{extendedchars=\true}

%межстрочный интервал
\usepackage{setspace}
\linespread{1.5}

% - выводим текст как есть с размером шрифта scriptsize
\makeatletter
\def\verbatim{\scriptsize\@verbatim \frenchspacing\@vobeyspaces \@xverbatim}
\makeatother
