Система управления манипулятором, как правило, имеет несколько уровней, каждый из которых может обслуживаться собственной микропроцессорной системой. Так, на уровне привода обеспечивается управление двигателем, осуществляющим движение одной или нескольких степеней подвижности. На следующем уровне системы управления манипулятором с помощью центрального процессора организуется координированная работа приводов манипулятора. При этом входной информацией является траектория, т. е. последовательность положений схвата манипулятора или связанного с ним объекта (инструмента, нагрузки).

Чтобы указать сервоприводу желаемое положение, по предназначенному для этого проводу необходимо посылать управляющий сигнал. Управляющий сигнал — импульсы постоянной частоты и переменной ширины.

То, какое положение должен занять сервопривод, зависит от длины импульсов. Когда сигнал поступает в управляющую схему, имеющийся в ней генератор импульсов производит свой импульс, длительность которого определяется через потенциометр. Другая часть схемы сравнивает длительность двух импульсов. Если длительность разная, включается электромотор. Направление вращения определяется тем, какой из импульсов короче. Если длины импульсов равны, электромотор останавливается. 

Для управления сервоприводами кухонного манипулятора использовалась плата Arduino Uno R3, с встроенной библиотекой «Servo» в которой по умолчанию выставлены следующие значения длин импульса: 544 мкс — для 0° и 2400 мкс — для 180°. 

Многие сервоприводы могут быть подключены к Arduino непосредственно. Для этого от них идёт шлейф из трёх проводов[3]:
\begin{itemize}
  \item красный — питание; подключается к контакту 5V или напрямую к источнику питания 
  \item коричневый или чёрный — земля
  \item жёлтый или белый — сигнал; подключается к цифровому выходу Arduino.
\end{itemize}

Все сервоприводы манипулятора были подключены непосредственно к плате Arduino Uno R3, которая в свою очередь при получении команды от ПК, подавала сигнал вращения на нужный угол, нужного сервопривода.
Библиотека Servo позволяет осуществлять программное управление сервоприводами. Для этого заводится переменная типа Servo. Управление осуществляется следующими функциями[4]:
\begin{itemize}
  \item attach() — присоединяет переменную к конкретному пину. Возможны два варианта синтаксиса для этой функции: servo.attach(pin) и servo.attach(pin, min, max). При  этом pin — номер пина, к которому присоединяют сервопривод, min и max — длины импульсов в микросекундах, отвечающих за углы поворота 0° и 180°. По умолчанию выставляются равными 544 мкс и 2400 мкс соответственно;
  \item write() — отдаёт команду сервоприводу принять некоторое значение параметра. Синтаксис следующий: servo.write(angle) , где angle — угол, на который должен повернуться сервопривод;
  \item writeMicroseconds() — отдаёт команду послать на сервопривод импульс определённой длины, является низкоуровневым аналогом предыдущей команды. Синтаксис следующий: servo.writeMicroseconds(uS) , где uS — длина импульса в микросекундах;
  \item read() — читает текущее значение угла, в котором находится сервопривод. Синтаксис следующий: servo.read(), возвращается целое значение от 0 до 180;
  \item attached() — проверка, была ли присоединена переменная к конкретному пину. Синтаксис следующий: servo.attached() , возвращается логическая истина, если переменная была присоединена к какому – либо пину, или ложь в обратном случае;
  \item detach() — производит действие, обратное действию attach() , то есть отсоединяет переменную от пина, к которому она была приписана. Синтаксис следующий: servo.detach().
\end{itemize}
 