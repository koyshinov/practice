\newpage

\begin{center} ИНДИВИДУАЛЬНОЕ ЗАДАНИЕ\end{center}

на производственную практику студенту Койшинову Тимуру Саматулы группы 743, факультета безопасности.

\begin{enumerate}
  \item Тема работы: Создание кухонного манипулятора
  \item Исходные данные к работе:
  \begin{enumerate}
    \item Данные по программированию на ARDUINO
    \item Данные по библиотеке OPENCV
  \end{enumerate}
  \item Срок сдачи студентом законченной работы \\ \underline{\hspace{17cm}}
  \item Содержание производственной практики:
  \begin{enumerate}
    \item Обзор манипулятора;
    \item Описание ARDUINO;
    \item Описание библиотеки компьютерного зрения OPENCV;
    \item Алгоритм работы программ:
    \begin{itemize}
      \item поиска предметов,
      \item управления манипулятором;
    \end{itemize}
	
  \end{enumerate}
  \item Содержание пояснительной записки:
  \begin{enumerate}
  \begin{itemize}
   	\item титульный лист;
   	\item задание;
    \item реферат;
   	\item содержание;
   	\item введение;
   	\item обзор манипулятора;
    \item обзор ARDUINO;
    \item обзор OPENCV;
    \item алгоритм работы программ;
    \item испытания; 
    \item заключение;
   	\item список использованных источников;
   	\item приложения.
  \end{itemize}
  \end{enumerate}

Отчет должен быть оформлена согласно ОС ТУСУР 01-2013.
	
	\item Дата выдачи задания: \underline{\hspace{8cm}}

\end{enumerate}

Задание согласовано:

Руководитель

\underline{\hspace{10cm}}

<<\underline{\hspace{1cm}}>>\underline{\hspace{3cm}} 2016г.
\underline{\hspace{4cm}}

Задание принято к исполнению

<<\underline{\hspace{1cm}}>>\underline{\hspace{3cm}} 2016г.
\underline{\hspace{4cm}}